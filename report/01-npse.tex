\documentclass[main.tex]{subfiles}
\begin{document}

\section{Non-parametric Spectrum Estimation}


\subsection{Discrete Fourier Transform Basis}

\subsubsection{Ideal Fourier Spectrum and DTFT}

Before diving into estimations with matlab, it's useful to be able to sketch out the expected results for the ideal Fourier specturm, as well as the DFT for a signal, and to compare this with the restuls we will gather in Matlab. To this purpose, a 20Hz signal is considered, and it's fourier spectrum is drawn side by side with the DFT (\textit{Using a windowed input}).

%TODO Sketch

It's important to note that the DFT will be symmetrical around the y-axis, and will be repeated every $2\pi$ \textit{Sampling in the time domain is repitition on the frequency domain. Sampling in the frequency domain is repitition in the time domain}.


\subsubsection{Comparison to Matlab result}

Turning to matlab, the results found above can be verified. By using the DFT with two different point-numbers, the two results found above are plotted.

\begin{figure}[H]
	\centering 
	\resizebox{0.8\textwidth}{!}{\input{matlabimages/1-1-b.tikz}}
	\caption{100 (left) and 1000 (right) point DFT results using matlab's {\tt fft} function. The {\tt fftshit} function has been used to center the results around the y-axis.}
	\label{fig:q1_1_b}
\end{figure}

Figure~\ref{fig:q1_1_b} clearly shows the effect of windowing on an otherwise perfect sinusoid. The left image shows two clean peaks at $20$ and $-20$ Hz, as expected. When increasing the FFT size for the same signal of 100 samples, the original signal is zero padded to the right to match the fft-size. Therefore, when taking a 1000-point FFT, the sinusoid is zero padded with 900 zeros. Conceptually, this is the same effect of the infinite sinuoid being windowed by a rectangular window over the first 100 samples.

\paragraph{As the input for the 1000-point FFT is windowed,} a number of lobes appear in the frequency spectra. This comes from two fact; 
\begin{enumerate}
	\item $rect(t) \Leftrightarrow sinc(\pi f)$, which will generate a main lobe and a number of side lobes when looking at the magnitude spectrum.
	\item Multiplication in the time domain is convolution in the frequency domain, therefore a sinc will appear centered at $\pm 20Hz$ when looking at the magnitude spectrum.
\end{enumerate}

It's imporant to note that the {\tt fft} function will only give a result over a single spectrum of $[-\pi,\pi)$ (or $[0, 2\pi)$ without {\tt fftshift}). However, the true DFT will be repeated every multiple of $2\pi$, such that $X(w) = X(w+2\pi)$.

\subsubsection{Incoherent Sampling}

Zero padding as in 1.1(b) is not the only scenario where the DFT of a single frequency wave turns into a fuller spectrum. Any discontinuity in the time domain will result in a fuller frequency response. This is the same effect as the windowing seen above. 

\paragraph{Discontinuous signals} will display similar properties as zero-padded signals. To overcome this, careful consideration of the time chunk passed through the FFT should be made. In reality, this is rarely possible, so a range of other windows are considered to smoothen out discontinuities. These are discussed in detail in this section.

Further to the discontinuity, which will explain why a full spectrum is to be seen in figure~\ref{fig:q1_1_c} as opposed to two dirac functions, taking the FFT over 100 points makes it hard to see one, single, clear peak. As the FFT with N points will have a frequency resolution of $f_s/N$, in the case where $N=100$, the resolution is 10Hz. Therefore, at $24Hz$ where a dirac is expected, there is no sample.

\paragraph{It is therefore important to ensure the fft size} gives the most appropriate frequency resolution required in each case. There is generally a tradeoff between the computational complexity of increasing the FFT size and the resulting frequency resolution, and so striking the correct balance is important.

\begin{figure}[H]
	\centering 
	\resizebox{0.7\textwidth}{!}{\input{matlabimages/1-1-c.tikz}}
	\caption{100 point DFT of a 24Hz wave}
	\label{fig:q1_1_c}
\end{figure}


\paragraph{To solve the incoherent sampling problem,} %TODO




















\subsection{Properties of Power Spectral Density (PSD)}

%TODO Proof

\subsubsection{Symmetry of input vector x}

With $\textbf{x}$ defined as $\textbf{x} = [r(0), r(1),...r(M-1),0,...0,r(-M-1),...,r(-1)]^T$, the time domain and the frequency domain for $\textbf{x}$ are shown. 

\begin{figure}[H]
	\centering 
	\resizebox{\textwidth}{!}{\input{matlabimages/1-2-a.tikz}}
	\caption{Plot of vector $\textbf{x}$ and corresponding FFT with L=256 and M=10,128.}
	\label{fig:q1_2_a}
\end{figure}


In reality, the PSD of an input is estiamted by the FFT of the autocovariance function $r(k)$. However, this signal is defined for negative values of $k$, as the signal is non-zero for $-M < k < 0$. These negative indecies are not handled well in matlab's {\tt fft()} function, and so to overcome this the input signal is passed through a circular shift\footnote{The circular shift in time will have no effect on the \textbf{magnitude} of the FFT, so is valud.} such that $x(L) = r(-1)$.

\begin{figure}[H]
	\centering 
	\resizebox{0.8\textwidth}{!}{\input{matlabimages/1-2-a-shift.tikz}}
	\caption{Circular shift of the autocorrelation function. Zero padding must be preserved to the right of the ACF, therefore this is circular.}
	\label{fig:q1_2_a-shift}
\end{figure}

As our correlation function might not be defined for all values of L, where L is the number of points for the FFT, zero-padding must be considered. Zero padding should be applied to the original auto-correlation function, and so this happens before the circular shift is applied. Therefore, after the circular shift, the 0 padding will appear in the \textbf{center} of the vector $\textbf{x}$, as opposed to the end of it.


\subsubsection{Removing Round-Off Errors}

As the autocorrelation function $\textbf{r(k)}$ is symmetrical, non-negative and real, it's expected that the FFT is real for all $R(\omega)$. However, due to rounding errors arising from floating-point precision issues, there may be a small imaginary component to the spectral estimate in the above case. 

A simple analysis of the value of $\textbf{fft\_x} = fft(\textbf{x})$ shows that all imaginary components are in the magnitude of $10^{-14}$, and can be safely ignored as $real(\textbf{fft\_x}) = abs(\textbf{fft\_x})$. 

%TODO how to demonstrate it's better?


\subsubsection{FFT of Asymmetrical Input Vector z}

Although $\textbf{z}$ is similar to $\textbf{x}$, the main difference is the order in which the shift and the zero padding have been applied. The fact that $\textbf{z}$ has zero padding  to $L$ applied \textbf{after} the shift into positive k values, the actual PSD we are estimating is of a signal with ACF shown below.

\begin{figure}[H]
	\centering 
	\resizebox{0.8\textwidth}{!}{\input{matlabimages/1-2-c.tikz}}
	\caption{Input vector $\textbf{z}$ shown on the left, with the equivalent ACF function on the right. Note the ACF function is not symmetrical, therefore cannot be accomplished by a real signal. The FFT will have imaginary components.}
	\label{fig:q1_2_c}
\end{figure}

If the time shifted (but not circularly-shifted) ACF funciton is considered, such that $\textbf{z}=[r(-M+1),...,r(-1),r(0),r(1),...,r(M-1),0,...,0]$, the imaginary part can no longer be safely ignored. As the input $\textbf{z}$ is no longer 
symmetrical, the fft is expected to have imaginary parts.

\subsubsection{Centering the DFT and ACF values around 0}

As Matlab works onyl with positive indexes, the use of matlab functions such as {\tt fft} will give the spectral estimation in the range $[0,2\pi)$. The function {\tt fftshift} is a helpful way to convert from $[0,2\pi)$ to the more conventional $[-\pi,\pi)$. This can then be plotted with matlab's plot function {\tt plot(x,fftshift(fft(x)))}, where x is a vector containing the values of $[-\pi,\pi)$.

This vector can be calculated by the value {\tt linspace(-$\pi$, $\pi$, L)} where L is the number of points in the FFT ({\tt length(fft(x))}). The reverse action, which is to annotate the correct values of time for the {\tt ifft}, the values of t should be {\tt linspace((L/2)-1, L/2, L)}. %TODO Go over this....




















\subsection{Resolution and Leakage of Periodogram-based Methods}


\subsubsection{Bartlett Window}

\begin{figure}[H]
	\centering 
	\resizebox{\textwidth}{!}{\input{matlabimages/1-3-a.tikz}}
	\caption{Linear and dB scale plots of a Bartlett window}
	\label{fig:q1_3_a}
\end{figure}

Using the inbuilt {\tt bartlett(n)} function, the {\tt fft} of the bartlett window with a number of different sizes are investigated to understand the properties of the window (see figure~\ref{fig:q1_3_a}). Specifically of interest is the $3dB$ width and the peaks (in dB) of the first sidelobe as functions of N.  

%TODO: Width and lobes as function of N




\subsubsection{Investigating window size for frequency estimation ($\alpha$)}

The ability to differentiate between two frequencies in the frequency domain is fundamental to signal processing. Window size plays a large role in this, and as seen above, the length of the window defines the ability to distinguish between signals in the frequency space.

To investigate this, the signal 

\begin{equation}
x(n) = a_1sin(f_02\pi n + \phi_1) + a_2sin((f_0+\alpha/N)2\pi n + \phi_2) + w(n)
\end{equation}

where w(n) is defined by $\mathcal{M}(0, \sigma^2)$ is considered. A number of alpha values are used, and resulting periodogram is plotted using the bartlett window, where $N=256$ in both cases. Figure~\ref{fig:q1_3_b} shows that from around an $\alpha$ value of 4, two distinct peaks can be seen in the periodogram, allowing for the two frequencies to be distinguished.

%TODO comment on this number

\begin{figure}[H]
	\centering 
	\resizebox{\textwidth}{!}{\input{matlabimages/1-3-b.tikz}}
	\caption{Effects of $\alpha$ on frequency resolution in the time domain.}
	\label{fig:q1_3_b}
\end{figure}




\subsubsection{Hamming-windowed periodogram}

As every window has different properties, it's important to realise that each window has it's own effect on the periodogram. The Hamming window has wider lobes when compared to the bartlett window. As the lobes are more spread out, the value of $\alpha$ required to distinguish frequencies is slightly higher. To achieve a similar frequency gap as shown in the bartlett window above, and $\alpha$ fo 4 should be chose.  %TODO: What frequency difference does this mean?

\begin{figure}[H]
	\centering 
	\resizebox{\textwidth}{!}{\input{matlabimages/1-3-c.tikz}}
	\caption{Effects of $\alpha$ on frequency resolution in the time domain when using the hamming window.}
	\label{fig:q1_3_c}
\end{figure}


\subsubsection{Effect of Lobe Leakage on the periodogram}

As well as the main lobes interfering with one another, as shown in section 1.3.2 and 1.3.3, the sidelobes can have an affect on frequency detection in the case where the amplitudes of the signals vary greatly. This is as the main lobe of a smaller amplitude signal might become hidden behind a side lobe of a large amplitude signal.

\begin{figure}[H]
	\centering 
	\resizebox{\textwidth}{!}{\input{matlabimages/1-3-d.tikz}}
	\caption{Effect of window leakage for a fixed alpha ($\alpha=4$) but varying signal amplitudes.}
	\label{fig:q1_3_d-4}
\end{figure}

By ranging over a number of magnitude for the signal $a_2$ (\textit{increasingly smaller}), this effect can be investigated. Here, the rectangular window is used. Figure~\ref{fig:q1_3_d-4} shows how, as the amplitude difference increases, the ability to distinguish frequencies diminishes.


\begin{figure}[H]
	\centering 
	\resizebox{\textwidth}{!}{\input{matlabimages/1-3-d-alpha-12.tikz}}
	\caption{Effect of window leakage for a fixed alpha ($\alpha=12$) but varying signal amplitudes.}
	\label{fig:q1_3_d-12}
\end{figure}

Comparing figure~\ref{fig:q1_3_d-4} with figure~\ref{fig:q1_3_d-12}, it can be seen that changing the frequency gap between the two sinusoids ($\alpha$) has only a limited effect on the lobe leakage effect. %TODO do more here.



\subsubsection{Fourier tranform of the Bartlett window}

%TODO: I don't understand this



\subsubsection{Chebyshev window and fixed sidelobe sizes}

In certain situations, such as when it is known that there will be large amplitude differences in input signals, it can be useful to decide the sidelobe level. A such filter is the Chebyshev filter, which can be instanciated in Matlab with {\tt chebyshev(N,U)}, with N the window size and U the sidelobe attenuation in decibels.

To explore the effect of the Chebyshev filter, two realisations of $x(n)$ with $a_1 =1$ and $a_2=[0.1, 0.0001]$, and the results in fig~\ref{fig:q1_3_f} show how the Chebyshev filter can distinguish two peaks as the value of $U$ increases.

\begin{figure}[H]
	\centering 
	\resizebox{\textwidth}{!}{\input{matlabimages/1-3-f.tikz}}
	\caption{Effect of the Chebyshev filter with different input signals and sidelobe attenuation. The difference of the two coefficients $a_1$ and $a_2$ is order 10, and 1000 for the second row.}
	\label{fig:q1_3_f}
\end{figure}

\paragraph{Reduced sidelobes do not come without a cost.} Note how the 3 dB point of the chebyshev filter increases as a function of N. Although the ability to differentiate signals with vastly different amplitude is improved, the tradeoff here lies between frequency or amplitude resolution.


The experiment run in section (d) is repeated with the chebyshev filters with a fixed sidelobe attenuation size to maxise the differentiation. As there is now a tradeoff between freqency and amplitude resolution, the value of $\alpha$ plays a crucial role. It is shown that for $\alpha = 4$, the chebyshev window is no better than the rectangular windows. However, when the frequency gap is increased, the chebyshev window is able to find two distinct peaks even at $a_1:a_2 = 1,000,000:1$.

Note how we are still able to identify both signals with $a_2:a_1$ at %TODO fill this and add something about alpha



















\subsection{Periodogram-based Methods Applied to Real-World Data}




\subsubsection{Sunspot Time Series}


\subsubsection{Electroencephalogram (EEG)}










\subsubsection{title}


\clearpage

\end{document}